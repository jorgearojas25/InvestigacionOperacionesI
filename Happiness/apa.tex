\documentclass[jou,apacite]{apa6}
\usepackage[utf8]{inputenc}
\usepackage{url}

\title{Alumnos entre el estres y la felicidad}
\shorttitle{APA style}

\author{Rojas Bautista Jorge Andrés}
\affiliation{Universidad Distrital Francisco Jose de Caldas}

\abstract{La relacion entre la felicidad y las aptitudes presentadas por los estudiantes estan fuertemente relacionadas, este estudio tiene como fin encontrar evidencias de como el desempeño academico puede varia entre el nivel de estres y de felicidad.}

\rightheader{Rojas Bautista Jorge Andres}
\leftheader{Alumnos entre el estres y la felicidad}

\begin{document}
\maketitle    
                        
\section{Introducción}
La autoeficacia es una construcción muy importante descubierta y desarrollada por el sociólogo Albert Bandura. Los altos niveles de autoeficacia ayudan a una persona a realizar con confianza y alcanzar sus objetivos. La influencia moderadora positiva de la autoeficacia se ha estudiado en muchos contextos. Los niveles de autoeficacia de una persona se ven influidos por diversos factores, como el rendimiento logrado, la experiencia vicaria, la persuasión verbal y los estados fisiológicos.~\cite{Andura}

Las expectativas de los empleadores de los graduados de los programas técnicos aumentan día a día debido a la intensa competencia en el mercado. Como resultado, las instituciones educativas han comenzado a ejercer más presión sobre los estudiantes al aumentar el rigor de sus programas. Esperan que hacerlo les permita manejar las expectativas del mundo corporativo y otros empleadores con mayor facilidad. Pero esta mayor expectativa de los estudiantes también ha dado lugar a un aumento en el estrés percibido por ellos. Esto es evidente por la creciente tasa de suicidios en las instituciones educativas de educación superior, la mayor necesidad de consejeros estudiantiles, el aumento de las enfermedades relacionadas con el estrés entre los estudiantes, etc.

La felicidad se ha definido como "un estado mental que resulta de un alto grado de satisfacción de las diversas necesidades y deseos en las dimensiones físicas, emocionales, cognitivas y espirituales de la vida y las oportunidades para el crecimiento personal, como resultado de lo cual uno se siente tranquilo, relajado y pacífico; mayormente satisfecho con la vida en la mayoría de sus aspectos y, sin embargo, ansioso por cumplir con los deberes de la mejor manera posible; amor y benevolencia para los demás; una parte significativa de un todo significativo más grande y la vida parece tener un significado y un propósito~\cite{chen}. No es un estado de ánimo, sino un estado mental que no se altera fácilmente por los altibajos ordinarios de la vida ".~\cite{Prasad}

El estrés entre los estudiantes de pregrado es multifactorial y surge de factores académicos y no académicos. El estrés entre los estudiantes aumenta durante las diferentes pruebas y el período de examen. Diferentes estudiantes no perciben el estrés de la misma manera a pesar de que el nivel de factores estresantes en el ambiente puede ser igual o igual.~\cite{Brand} Los estudiantes jóvenes se ven más afectados por el estrés cuando el entorno es más competitivo


\section{Efectos de la felicidad desborada}
Los efectos de la autoeficacia académica como el optimismo en el rendimiento académico del estudiante, el estrés, la salud y el compromiso de permanecer en la escuela. Los estudiantes que ingresan a la universidad con confianza en su capacidad para desempeñarse bien académicamente tienen un rendimiento significativamente mejor que los estudiantes menos seguros. Se estudiaron las relaciones entre las variables predictoras (promedio de calificaciones de la escuela secundaria, autoeficacia académica y optimismo) y las variables del moderador (expectativas académicas y capacidad de afrontamiento autopercibida) y se encontró que es probable que los estudiantes confiados y optimistas den como resultado ajuste exitoso.~\cite{Martin} Dichos estudiantes tienen mayores expectativas para ellos mismos en parte porque confían en sus capacidades y ven el mundo como menos amenazante, y su capacidad para responder a él tan alto.


Se supone que la felicidad es el objetivo final de la vida y, por lo tanto, se considera muy importante en la vida. Cualquier acción que realice el ser humano es, en última instancia, ser feliz. Aristóteles lo llamó el summum bonnum o bien principal en la vida. En los últimos tiempos las personas no se han vuelto más felices que las generaciones anteriores a pesar de que hay una mejora drástica en la tecnología y los estándares de estilo de vida.~\cite{Vasudha} El aumento del narcisismo tal como lo descubrieron algunos investigadores puede ser una de las causas de esto. La necesidad de la autorrealización, que constituye la cúspide de la famosa teoría de la jerarquía de las necesidades propuesta por Maslow, se supone que hace que una persona esté muy arraigada y feliz al centrarse en el cumplimiento de su potencial frente a la aprobación social que anhelan los narcisistas~\cite{Maslow}. Una mejor comprensión de la relación entre estos tres constructos es extremadamente importante, ya que sería beneficioso tanto para la teoría (en el sentido de que promovería una mejor comprensión de la autorrealización, la felicidad y el narcisismo como la relación entre estas variables), y para la práctica aplicaciones. Si es probable que una mayor autorrealización reduzca el narcisismo, lo que a su vez aumentará la felicidad con la vida y el bienestar subjetivo, sería aconsejable que nuestros lugares de trabajo propicien mayores grados de autorrealización~\cite{Rose}.

Todo un campo de investigación se ha desarrollado en torno al concepto más inclusivo de bienestar subjetivo, que se caracteriza por una amplia colección de fenómenos relacionados con la felicidad en lugar de una emoción momentánea específica. Como se podría esperar, las personas que son felices de esta manera tienden a experimentar emociones positivas frecuentes y emociones negativas poco frecuentes. Sin embargo, esta forma más amplia de felicidad no es puramente emocional: también tiene un componente cognitivo. Cuando a las personas felices se les pide que reflexionen sobre las condiciones y los eventos en sus vidas, tienden a evaluar estas condiciones y eventos positivamente~\cite{Happi}. Por lo tanto, las personas felices informan estar satisfechos con sus vidas y los diversos dominios en sus vidas.


\section{Riesgos del estres academico}

Una persona con un nivel de autoeficacia más alto experimentará una menor cantidad de estrés percibido. En un programa de postgrado en la misma universidad, todos trabajan bajo los mismos niveles de estrés. Por lo tanto, depende completamente del temperamento de los estudiantes individuales en cuanto a cómo perciben el estrés. Cuando un estudiante tiene un nivel de autoeficacia más alto, tiene más confianza para cumplir con las contingencias ambientales y, por lo tanto, percibe menos estrés en la misma situación en comparación con alguien con un nivel de autoeficacia más bajo.


El apoyo social se define como la percepción o experiencia de que uno es amado y cuidado, estimado y valorado, y parte de una red social de asistencia mutua y obligaciones.~\cite{Dunkel} Actualmente se puede dividir en dos categorías: una es el apoyo social aprobado que es independiente de la cognición individual y existe objetivamente, incluida la asistencia directa con la sustancia y la red social cuando el individuo está en situaciones de estrés; El otro es el apoyo social percibido que se refiere al juicio subjetivo del receptor de que los proveedores ofrecerán (o han ofrecido) ayuda efectiva en momentos de necesidad.

Por ejemplo un estudio en universidades de China donde se usó la Escala general de autoeficacia (Schwarzer, R., and Jerusalem, M. 1995) para medir los niveles de autoeficacia de un alumno~\cite{Schwarzer}. Perceived Stress Scale (Sheldon Cohen, 1994) se usó para medir los niveles de estrés entre los estudiantes del programa de pregrado. conluyo que el personal de la universidad enfatiza que ya está relacionado con todo el entorno social, también relacionado con la cultura escolar y la organización, pero también relacionado con el personal. El estrés entre el hombre y la mujer tiene la ligera diferencia, explicó que el personal masculino asumió la responsabilidad en el trabajo y la familia, esto puede deberse a diferentes roles y responsabilidades en la sociedad como resultado de diferentes expectativas culturales y sociales y ambientales factores. Con el fin de realizar un análisis para ver qué personal femenino y masculino informaron los factores estresantes principales, se calculó la media de factores estresantes, por medio de los factores estresantes, los 5 principales para mujeres son el pago incompleto; Demasiada burocracia; La falta de oportunidades para tener intercambio académico; Mantener exceso de personal en detestación; El salario insuficiente y las bonificaciones para cubrir el consumo diario y para hombres son pagos incompletos; Mantener exceso de personal en detestación; Demasiada burocracia; La falta de oportunidades para tener intercambio académico; Tener un tiempo inadecuado para profundizar la investigación profesional.~\cite{Liu}

La ansiedad social es una especie de autoconciencia de inquietud frente a los demás~\cite{Fenistein}. Predomina principalmente en adolescentes de 13 a 24 años, pero rara vez después de los 25 años [2]. Existen diferencias culturales que existen en la ansiedad social. Es más popular en adolescentes asiáticos ~\cite{Kan}

\section{Confrontación}

El estrés está aumentando en las diferentes instituciones académicas, lo que se refleja en casos frecuentes de suicidios estudiantiles, depresión, necesidad de psicólogos expertos para actuar como consejeros estudiantiles, etc. En este contexto, este estudio es una adición muy útil al campo. Se exploró la relación entre la autoeficacia, el estrés percibido y la felicidad estudiantil de los estudiantes de una facultad de ingeniería con un ambiente académico desafiante. Los resultados son bastante prometedores~\cite{Shilpa}. La autoeficacia se ha demostrado en este estudio como un importante y fuerte predictor de la felicidad del estudiante. Un mayor nivel de autoeficacia también ayuda a reducir el estrés percibido. Merece ser estudiado a una escala mucho mayor en diferentes entornos académicos e industriales donde los niveles de estrés son muy altos. Es especialmente pertinente realizar tales estudios porque ha habido una amplia investigación en el área de la autoeficacia que ha demostrado que se puede aumentar en personas con el tipo de entorno e intervenciones adecuadas. Los colegios y universidades de todo el mundo pueden pensar en estrategias de intervención destinadas a aumentar la autoeficacia general de sus estudiantes.

La sociedad moderna está llena de estrés. Numerosas personas viven sus vidas con una variedad de factores estresantes. El estrés es una reacción biológica que se desarrolla cuando enfrentamos un factor estresante psicológico o espiritual~\cite{Yata}. Los procesos reactivos de las personas que interactúan con el entorno significan procesos cognitivos individuales involucrados en reacciones biológicas y procesos fisiológicos. Existen tres índices de cuerpos biológicos, psicología y acción que exigen la consideración de procesos individuales cuando están sometidos a estrés. Además, estamos dirigidos a un índice psicológico que muestra el estrés como psicológico. En general, aparecen varios síntomas somáticos con gastralgia y estados sin aliento cuando se acumula estrés. El estrés aparece desde un estado psicológico cuando uno experimenta frustración y malestar. Además, el estrés se manifiesta en acciones como el aumento en el consumo de alcohol y cigarrillos. Además, el estrés afecta el cerebro. Por lo general, el cerebro humano responde eficazmente para mantener el equilibrio mental y físico. Sin embargo, el estrés excesivo puede desencadenar enfermedades mentales como la depresión ~\cite{catherine}. Debido a que puede ser difícil ocultar el estrés, las caras a menudo se llaman una ventana por la cual uno puede discernir información de varios tipos, como la modalidad de la mente de una persona y el estado de salud. Especialmente, las expresiones faciales pueden mostrar aspectos de la psicología interna, emociones reflexivas como el deleite, la ira, el dolor, el placer y la existencia de estrés~\cite{Sato}. Los amigos cercanos y los miembros de la familia se comunican al interpretar el estrés de las condiciones y los cambios de las expresiones faciales.


Estos hallazgos sugieren que la autorrealización es una construcción muy útil cuando se trata de verificar el nivel de narcisismo que tiende a aumentar estos días con las plataformas de Internet y redes sociales que aumentan las posibilidades de que una persona sea muy visible frente a grandes audiencias Es una comprensión muy importante porque los psicólogos saben que el narcisismo es una cualidad no saludable.~\cite{Schmidt} Ha sido identificado como un trastorno de la personalidad por los psiquiatras y genera frustración crónica e infelicidad hacia uno mismo y hacia las personas con las que uno está relacionado. La autorrealización por otro lado mejora la salud y la felicidad psicológicas.~\cite{Raskin} Los gobiernos y las organizaciones, así como las familias, deberían tratar de comprender qué es la autorrealización y cómo promoverla entre más y más personas bajo su influencia. Sería muy útil para restringir las tendencias narcisistas y maximizar la felicidad con la vida.




\bibliography{Biblio}

\end{document}
