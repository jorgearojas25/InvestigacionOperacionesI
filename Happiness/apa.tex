\documentclass[jou,apacite]{apa6}
\usepackage[utf8]{inputenc}

\title{Alumnos entre el estres y la felicidad}
\shorttitle{APA style}

\author{Rojas Bautista Jorge Andrés}
\affiliation{Universidad Distrital Francisco Jose de Caldas}

\abstract{La relacion entre la felicidad y las aptitudes presentadas por los estudiantes estan fuertemente relacionadas, este estudio tiene como fin encontrar evidencias de como el desempeño academico puede varia entre el nivel de estres y de felicidad.}

\rightheader{APA style}
\leftheader{Author One}

\begin{document}
\maketitle    
                        
\section{Introducción}
La autoeficacia es una construcción muy importante descubierta y desarrollada por el sociólogo Albert Bandura. Los altos niveles de autoeficacia ayudan a una persona a realizar con confianza y alcanzar sus objetivos. La influencia moderadora positiva de la autoeficacia se ha estudiado en muchos contextos. Los niveles de autoeficacia de una persona se ven influidos por diversos factores, como el rendimiento logrado, la experiencia vicaria, la persuasión verbal y los estados fisiológicos.~\cite{Andura}

Las expectativas de los empleadores de los graduados de los programas técnicos aumentan día a día debido a la intensa competencia en el mercado. Como resultado, las instituciones educativas han comenzado a ejercer más presión sobre los estudiantes al aumentar el rigor de sus programas. Esperan que hacerlo les permita manejar las expectativas del mundo corporativo y otros empleadores con mayor facilidad. Pero esta mayor expectativa de los estudiantes también ha dado lugar a un aumento en el estrés percibido por ellos. Esto es evidente por la creciente tasa de suicidios en las instituciones educativas de educación superior, la mayor necesidad de consejeros estudiantiles, el aumento de las enfermedades relacionadas con el estrés entre los estudiantes, etc.

La felicidad se ha definido como "un estado mental que resulta de un alto grado de satisfacción de las diversas necesidades y deseos en las dimensiones físicas, emocionales, cognitivas y espirituales de la vida y las oportunidades para el crecimiento personal, como resultado de lo cual uno se siente tranquilo, relajado y pacífico; mayormente satisfecho con la vida en la mayoría de sus aspectos y, sin embargo, ansioso por cumplir con los deberes de la mejor manera posible; amor y benevolencia para los demás; una parte significativa de un todo significativo más grande y la vida parece tener un significado y un propósito. No es un estado de ánimo, sino un estado mental que no se altera fácilmente por los altibajos ordinarios de la vida ".~\cite{Prasad}

El estrés entre los estudiantes de pregrado es multifactorial y surge de factores académicos y no académicos. El estrés entre los estudiantes aumenta durante las diferentes pruebas y el período de examen. Diferentes estudiantes no perciben el estrés de la misma manera a pesar de que el nivel de factores estresantes en el ambiente puede ser igual o igual. Los estudiantes jóvenes se ven más afectados por el estrés cuando el entorno es más competitivo~\cite{Brand}


\subsection{Evidencias}


\bibliography{Biblio}

\end{document}
